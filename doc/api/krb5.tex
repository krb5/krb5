The main functions deal with the nitty-gritty details: verifying
tickets, creating authenticators, and the like.

\begin{funcdecl}[krb5_encode_kdc_rep]{krb5_error_code}{\funcin}
\funcarg{krb5_msgtype}{type}
\funcarg{krb5_enc_kdc_rep_part *}{encpart}
\funcarg{krb5_keyblock *}{client_key}
\funcinout
\funcarg{krb5_kdc_rep *}{dec_rep}
\funcout
\funcarg{krb5_data *}{enc_rep}
\end{funcdecl}

Takes KDC rep parts in \funcparam{*rep} and \funcparam{*encpart}, and
formats it into \funcparam{*enc_rep}, using message type \funcparam{type}
and encryption key \funcparam{client_key} and encryption type
\funcparam{dec_rep{\ptsto}etype}.

\funcparam{enc_rep{\ptsto}data} will point to  allocated storage upon
non-error return; the caller should free it when finished.

Returns system errors.

\begin{funcdecl}[krb5_decode_kdc_rep]{krb5_error_code}{\funcin}
\funcarg{krb5_data *}{enc_rep}
\funcarg{krb5_keyblock *}{key}
\funcarg{krb5_enctype}{etype}
\funcout
\funcarg{krb5_kdc_rep **}{dec_rep}
\end{funcdecl}

Takes a KDC_REP message and decrypts encrypted part using
\funcparam{etype} and \funcparam{*key}, putting result in \funcparam{*rep}.
The pointers in \funcparam{dec_rep}
are all set to allocated storage which should be freed by the caller
when finished with the response (by using \funcname{krb5_free_kdc_rep}).


If the response isn't a KDC_REP (tgs or as), it returns an error from
the decoding routines (usually ISODE_50_LOCAL_ERR_BADDECODE).

Returns errors from encryption routines, system errors.

