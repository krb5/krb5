The main functions deal with the nitty-gritty details: verifying
tickets, creating authenticators, and the like.

\begin{funcdecl}{krb5_encode_kdc_rep}{krb5_error_code}{\funcin}
\funcarg{krb5_msgtype}{type}
\funcarg{krb5_enc_kdc_rep_part *}{encpart}
\funcarg{krb5_keyblock *}{client_key}
\funcinout
\funcarg{krb5_kdc_rep *}{dec_rep}
\funcout
\funcarg{krb5_data *}{enc_rep}
\end{funcdecl}

Takes KDC rep parts in \funcparam{*rep} and \funcparam{*encpart}, and
formats it into \funcparam{*enc_rep}, using message type \funcparam{type}
and encryption key \funcparam{client_key} and encryption type
\funcparam{dec_rep{\ptsto}etype}.

\funcparam{enc_rep{\ptsto}data} will point to  allocated storage upon
non-error return; the caller should free it when finished.

Returns system errors.

\begin{funcdecl}{krb5_decode_kdc_rep}{krb5_error_code}{\funcin}
\funcarg{krb5_data *}{enc_rep}
\funcarg{krb5_keyblock *}{key}
\funcarg{krb5_enctype}{etype}
\funcout
\funcarg{krb5_kdc_rep **}{dec_rep}
\end{funcdecl}

Takes a KDC_REP message and decrypts encrypted part using
\funcparam{etype} and \funcparam{*key}, putting result in \funcparam{*rep}.
The pointers in \funcparam{dec_rep}
are all set to allocated storage which should be freed by the caller
when finished with the response (by using \funcname{krb5_free_kdc_rep}).


If the response isn't a KDC_REP (tgs or as), it returns an error from
the decoding routines (usually ISODE_50_LOCAL_ERR_BADDECODE).

Returns errors from encryption routines, system errors.

\begin{funcdecl}{krb5_kdc_rep_decrypt_proc}{\funcin}
\funcarg{krb5_keyblock *}{key}
\funcarg{krb5_pointer}{decryptarg}
\funcinout
\funcarg{krb5_kdc_rep *}{dec_rep}
\end{funcdecl}
Decrypt the encrypted portion of \funcparam{dec_rep}, using the
encryption key \funcparam{key}.

The result is in allocated storage pointed to by
\funcparam{dec_rep{\ptsto}enc_part2}, unless some error occurs.

\begin{funcdecl}{krb5_encode_ticket}{krb5_error_code}{\funcin}
\funcarg{krb5_ticket *}{dec_ticket}
\funcout
\funcarg{krb5_data **}{enc_ticket}
\end{funcdecl}

Takes \funcparam{dec_ticket} (with associated encrypted part
\funcparam{dec_ticket{\ptsto}enc_part}),
and encodes for transmission, placing result in \funcparam{*enc_ticket}.
The string \funcparam{*enc_ticket} will be allocated before formatting.

Returns errors from encryption routines, system errors.

\begin{funcdecl}{krb5_decode_ticket}{krb5_error_code}{\funcin}
\funcarg{krb5_data *}{enc_ticket}
\funcout
\funcarg{krb5_ticket **}{dec_ticket}
\end{funcdecl}

Decodes formatted ticket \funcparam{enc_ticket},
filling in \funcparam{*dec_ticket} with a pointer to the results.
\funcparam{*dec_ticket} is set to allocated storage which should be
freed by the caller (by using \funcname{krb5_free_ticket}) when finished with
the ticket.

Returns system errors.


\begin{funcdecl}{krb5_encrypt_tkt_part}{krb5_error_code}{ \funcin}
\funcarg{krb5_keyblock *}{srv_key}
\funcinout
\funcarg{krb5_ticket *}{dec_ticket}
\end{funcdecl}

Takes unencrypted \funcparam{dec_ticket} and
\funcparam{dec_ticket{\ptsto}enc_part2}, encrypts with
\funcparam{dec_ticket{\ptsto}etype}
using \funcparam{srv_key}, and places result in
\funcparam{dec_ticket{\ptsto}enc_part}.
The string \funcparam{dec_ticket{\ptsto}enc_part} will be allocated
before formatting.

Returns errors from encryption routines, system errors

\funcparam{enc_part{\ptsto}data} is allocated and filled in with
encrypted stuff.

\begin{funcdecl}{krb5_decrypt_tkt_part}{krb5_error_code}{\funcin}
\funcarg{krb5_keyblock *}{srv_key}
\funcinout
\funcarg{krb5_ticket *}{dec_ticket}
\end{funcdecl}

Takes encrypted \funcparam{dec_ticket{\ptsto}enc_part}, encrypts with
\funcparam{dec_ticket{\ptsto}etype}
using \funcparam{srv_key}, and places result in
\funcparam{dec_ticket{\ptsto}enc_part2}.  The storage of
\funcparam{dec_ticket{\ptsto}enc_part2} will be allocated before return.

Returns errors from encryption routines, system errors

\begin{funcdecl}{krb5_send_tgs}{krb5_error_code}{\funcin}
\funcarg{krb5_flags}{options}
\funcarg{krb5_ticket_times *}{timestruct}
\funcarg{krb5_enctype}{etype}
\funcarg{krb5_cksumtype}{sumtype}
\funcarg{krb5_principal}{sname}
\funcarg{krb5_address **}{addrs}
\funcarg{krb5_authdata **}{authorization_data}
\funcarg{krb5_data *}{second_ticket}
\funcinout
\funcarg{krb5_creds *}{usecred}
\funcout
\funcarg{krb5_response *}{rep}
\end{funcdecl}

Sends a request to the TGS and waits for a response.
\funcparam{options} is used for the options in the KRB_TGS_REQ.
\funcparam{timestruct} values are used for from, till, and rtime in the
KRB_TGS_REQ.
\funcparam{etype} is used for etype in the KRB_TGS_REQ.
\funcparam{sumtype} is used for the checksum in the AP_REQ in the KRB_TGS_REQ
\funcparam{sname} is used for sname in the KRB_TGS_REQ.
\funcparam{addrs}, if non-NULL, is used for addresses in the KRB_TGS_REQ.
\funcparam{authorization_dat}, if non-NULL, is used for authorization_dat in the KRB_TGS_REQ.
\funcparam{second_ticket}, if required by options, is used for the 2nd
ticket in the KRB_TGS_REQ.
\funcparam{usecred} is used for the ticket and session key in the KRB_AP_REQ header in the KRB_TGS_REQ.

The KDC realm is extracted from \funcparam{usecred{\ptsto}server}'s realm.

The response is placed into \funcparam{*rep}.
\funcparam{rep{\ptsto}response.data} is set to point at allocated storage
which should be freed by the caller when finished.

Returns system errors.

\begin{funcdecl}{krb5_get_cred_from_kdc}{krb5_error_code}{\funcin}
\funcarg{krb5_ccache}{ccache}
\funcinout
\funcarg{krb5_creds *}{creds}
\funcout			
\funcparam{krb5_creds ***}{tgts }
\end{funcdecl}

Retrieve credentials for principal \funcparam{creds{\ptsto}client},
server \funcparam{creds{\ptsto}server},
ticket flags \funcparam{creds{\ptsto}ticket_flags}, possibly
\funcparam{creds{\ptsto}second_ticket} if needed by the ticket flags.

\funcparam{ccache} is used to fetch initial TGT's to start the authentication
path to the server.

Credentials are requested from the KDC for the server's realm.  Any
TGT credentials obtained in the process of contacting the KDC are
returned in an array of credentials; \funcparam{tgts} is filled in to
point to an array of pointers to credential structures (if no TGT's were
used, the pointer is zeroed).  TGT's may be returned even if no useful
end ticket was obtained.

The returned credentials are NOT cached.

If credentials are obtained, \funcparam{creds} is filled in with the results;
\funcparam{creds{\ptsto}ticket} and
\funcparam{creds{\ptsto}keyblock{\ptsto}key} are set to allocated storage,
which should be freed by the caller when finished.

Returns errors, system errors.


\begin{funcdecl}{krb5_free_tgt_creds}{void}{\funcin}
\funcarg{krb5_creds **}{tgts}
\end{funcdecl}

Frees the TGT credentials \funcparam{tgts} returned by
\funcname{krb5_get_cred_from_kdc}.

\begin{funcdecl}{krb5_get_credentials}{krb5_error_code}{\funcin}
\funcarg{krb5_flags}{options}
\funcarg{krb5_ccache}{ccache}
\funcinout
\funcarg{krb5_creds *}{creds}
\end{funcdecl}

Attempts to use the credentials cache \funcparam{ccache} or a TGS
exchange to get an additional ticket for the client identified by
\funcparam{creds{\ptsto}client}, the server identified by
\funcparam{creds{\ptsto}server}, with options \funcparam{options},
expiration date specified in \funcparam{creds{\ptsto}times.endtime} (0
means as long as possible), session key type specified in
\funcparam{creds{\ptsto}keyblock.keytype} (if non-zero).

Any returned ticket and intermediate ticket-granting tickets are
stored in \funcparam{ccache}.

Returns errors from encryption routines, system errors.



