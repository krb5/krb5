\documentstyle[newcen,fixunder,functions,changebar,twoside,fancyheadings,backslash]{article}
\setlength{\oddsidemargin}{0.25in}
\setlength{\evensidemargin}{-0.25in}
\setlength{\topmargin}{-.5in}
\setlength{\textheight}{9in}
\setlength{\parskip}{.1in}
\setlength{\parindent}{2em}
\setlength{\textwidth}{6.25in}
\makeindex
\newif\ifdraft
\drafttrue
\typein{Draft flag? (type \backslash draftfalse<CR> if not draft...)}
\ifdraft
\pagestyle{fancy}
\lhead[\fancyplain{}\thepage]{\fancyplain{}{\sl \leftmark}}
\rhead[\fancyplain{}{\sl \leftmark}]{\fancyplain{}\thepage}
\cfoot{{\bf DRAFT---DO NOT REDISTRIBUTE}}
\else\pagestyle{headings}\fi
\begin{document}
\thispagestyle{empty}
\begin{center}
{\Huge Kerberos V5 application programming library}
\ifdraft \\ {\Large DRAFT---\today}\fi
\end{center}
\section{libkrb5.a functions}
This section describes the functions provided in the \libname{libkrb5.a}
library.  The library is built from several pieces, mostly for convenience in
programming, maintenance, and porting.

\ifdraft\sloppy\fi

\subsection{Main functions}
The main functions deal with the nitty-gritty details: verifying
tickets, creating authenticators, and the like.

\begin{funcdecl}{krb5_encode_kdc_rep}{krb5_error_code}{\funcin}
\funcarg{krb5_msgtype}{type}
\funcarg{krb5_enc_kdc_rep_part *}{encpart}
\funcarg{krb5_keyblock *}{client_key}
\funcinout
\funcarg{krb5_kdc_rep *}{dec_rep}
\funcout
\funcarg{krb5_data *}{enc_rep}
\end{funcdecl}

Takes KDC rep parts in \funcparam{*rep} and \funcparam{*encpart}, and
formats it into \funcparam{*enc_rep}, using message type \funcparam{type}
and encryption key \funcparam{client_key} and encryption type
\funcparam{dec_rep{\ptsto}etype}.

\funcparam{enc_rep{\ptsto}data} will point to  allocated storage upon
non-error return; the caller should free it when finished.

Returns system errors.

\begin{funcdecl}{krb5_decode_kdc_rep}{krb5_error_code}{\funcin}
\funcarg{krb5_data *}{enc_rep}
\funcarg{krb5_keyblock *}{key}
\funcarg{krb5_enctype}{etype}
\funcout
\funcarg{krb5_kdc_rep **}{dec_rep}
\end{funcdecl}

Takes a KDC_REP message and decrypts encrypted part using
\funcparam{etype} and \funcparam{*key}, putting result in \funcparam{*rep}.
The pointers in \funcparam{dec_rep}
are all set to allocated storage which should be freed by the caller
when finished with the response (by using \funcname{krb5_free_kdc_rep}).


If the response isn't a KDC_REP (tgs or as), it returns an error from
the decoding routines (usually ISODE_50_LOCAL_ERR_BADDECODE).

Returns errors from encryption routines, system errors.

\begin{funcdecl}{krb5_kdc_rep_decrypt_proc}{\funcin}
\funcarg{krb5_keyblock *}{key}
\funcarg{krb5_pointer}{decryptarg}
\funcinout
\funcarg{krb5_kdc_rep *}{dec_rep}
\end{funcdecl}
Decrypt the encrypted portion of \funcparam{dec_rep}, using the
encryption key \funcparam{key}.

The result is in allocated storage pointed to by
\funcparam{dec_rep{\ptsto}enc_part2}, unless some error occurs.

\begin{funcdecl}{krb5_encode_ticket}{krb5_error_code}{\funcin}
\funcarg{krb5_ticket *}{dec_ticket}
\funcout
\funcarg{krb5_data **}{enc_ticket}
\end{funcdecl}

Takes \funcparam{dec_ticket} (with associated encrypted part
\funcparam{dec_ticket{\ptsto}enc_part}),
and encodes for transmission, placing result in \funcparam{*enc_ticket}.
The string \funcparam{*enc_ticket} will be allocated before formatting.

Returns errors from encryption routines, system errors.

\begin{funcdecl}{krb5_decode_ticket}{krb5_error_code}{\funcin}
\funcarg{krb5_data *}{enc_ticket}
\funcout
\funcarg{krb5_ticket **}{dec_ticket}
\end{funcdecl}

Decodes formatted ticket \funcparam{enc_ticket},
filling in \funcparam{*dec_ticket} with a pointer to the results.
\funcparam{*dec_ticket} is set to allocated storage which should be
freed by the caller (by using \funcname{krb5_free_ticket}) when finished with
the ticket.

Returns system errors.


\begin{funcdecl}{krb5_encrypt_tkt_part}{krb5_error_code}{ \funcin}
\funcarg{krb5_keyblock *}{srv_key}
\funcinout
\funcarg{krb5_ticket *}{dec_ticket}
\end{funcdecl}

Takes unencrypted \funcparam{dec_ticket} and
\funcparam{dec_ticket{\ptsto}enc_part2}, encrypts with
\funcparam{dec_ticket{\ptsto}etype}
using \funcparam{srv_key}, and places result in
\funcparam{dec_ticket{\ptsto}enc_part}.
The string \funcparam{dec_ticket{\ptsto}enc_part} will be allocated
before formatting.

Returns errors from encryption routines, system errors

\funcparam{enc_part{\ptsto}data} is allocated and filled in with
encrypted stuff.

\begin{funcdecl}{krb5_decrypt_tkt_part}{krb5_error_code}{\funcin}
\funcarg{krb5_keyblock *}{srv_key}
\funcinout
\funcarg{krb5_ticket *}{dec_ticket}
\end{funcdecl}

Takes encrypted \funcparam{dec_ticket{\ptsto}enc_part}, encrypts with
\funcparam{dec_ticket{\ptsto}etype}
using \funcparam{srv_key}, and places result in
\funcparam{dec_ticket{\ptsto}enc_part2}.  The storage of
\funcparam{dec_ticket{\ptsto}enc_part2} will be allocated before return.

Returns errors from encryption routines, system errors

\begin{funcdecl}{krb5_send_tgs}{krb5_error_code}{\funcin}
\funcarg{krb5_flags}{options}
\funcarg{krb5_ticket_times *}{timestruct}
\funcarg{krb5_enctype}{etype}
\funcarg{krb5_cksumtype}{sumtype}
\funcarg{krb5_principal}{sname}
\funcarg{krb5_address **}{addrs}
\funcarg{krb5_authdata **}{authorization_data}
\funcarg{krb5_data *}{second_ticket}
\funcinout
\funcarg{krb5_creds *}{usecred}
\funcout
\funcarg{krb5_response *}{rep}
\end{funcdecl}

Sends a request to the TGS and waits for a response.
\funcparam{options} is used for the options in the KRB_TGS_REQ.
\funcparam{timestruct} values are used for from, till, and rtime in the
KRB_TGS_REQ.
\funcparam{etype} is used for etype in the KRB_TGS_REQ.
\funcparam{sumtype} is used for the checksum in the AP_REQ in the KRB_TGS_REQ
\funcparam{sname} is used for sname in the KRB_TGS_REQ.
\funcparam{addrs}, if non-NULL, is used for addresses in the KRB_TGS_REQ.
\funcparam{authorization_dat}, if non-NULL, is used for authorization_dat in the KRB_TGS_REQ.
\funcparam{second_ticket}, if required by options, is used for the 2nd
ticket in the KRB_TGS_REQ.
\funcparam{usecred} is used for the ticket and session key in the KRB_AP_REQ header in the KRB_TGS_REQ.

The KDC realm is extracted from \funcparam{usecred{\ptsto}server}'s realm.

The response is placed into \funcparam{*rep}.
\funcparam{rep{\ptsto}response.data} is set to point at allocated storage
which should be freed by the caller when finished.

Returns system errors.

\begin{funcdecl}{krb5_get_cred_from_kdc}{krb5_error_code}{\funcin}
\funcarg{krb5_ccache}{ccache}
\funcinout
\funcarg{krb5_creds *}{creds}
\funcout			
\funcparam{krb5_creds ***}{tgts }
\end{funcdecl}

Retrieve credentials for principal \funcparam{creds{\ptsto}client},
server \funcparam{creds{\ptsto}server},
ticket flags \funcparam{creds{\ptsto}ticket_flags}, possibly
\funcparam{creds{\ptsto}second_ticket} if needed by the ticket flags.

\funcparam{ccache} is used to fetch initial TGT's to start the authentication
path to the server.

Credentials are requested from the KDC for the server's realm.  Any
TGT credentials obtained in the process of contacting the KDC are
returned in an array of credentials; \funcparam{tgts} is filled in to
point to an array of pointers to credential structures (if no TGT's were
used, the pointer is zeroed).  TGT's may be returned even if no useful
end ticket was obtained.

The returned credentials are NOT cached.

If credentials are obtained, \funcparam{creds} is filled in with the results;
\funcparam{creds{\ptsto}ticket} and
\funcparam{creds{\ptsto}keyblock{\ptsto}key} are set to allocated storage,
which should be freed by the caller when finished.

Returns errors, system errors.


\begin{funcdecl}{krb5_free_tgt_creds}{void}{\funcin}
\funcarg{krb5_creds **}{tgts}
\end{funcdecl}

Frees the TGT credentials \funcparam{tgts} returned by
\funcname{krb5_get_cred_from_kdc}.

\begin{funcdecl}{krb5_get_credentials}{krb5_error_code}{\funcin}
\funcarg{krb5_flags}{options}
\funcarg{krb5_ccache}{ccache}
\funcinout
\funcarg{krb5_creds *}{creds}
\end{funcdecl}

Attempts to use the credentials cache \funcparam{ccache} or a TGS
exchange to get an additional ticket for the client identified by
\funcparam{creds{\ptsto}client}, the server identified by
\funcparam{creds{\ptsto}server}, with options \funcparam{options},
expiration date specified in \funcparam{creds{\ptsto}times.endtime} (0
means as long as possible), session key type specified in
\funcparam{creds{\ptsto}keyblock.keytype} (if non-zero).

Any returned ticket and intermediate ticket-granting tickets are
stored in \funcparam{ccache}.

Returns errors from encryption routines, system errors.

\begin{funcdecl}{krb5_get_in_tkt}{krb5_error_code}{\funcin}
\funcarg{krb5_flags}{options}
\funcarg{krb5_address **}{addrs}
\funcarg{krb5_enctype}{etype}
\funcarg{krb5_keytype}{keytype}
\funcfuncarg{krb5_error_code}{(*key_proc)}
	\funcarg{krb5_keytype}{type}
	\funcarg{krb5_keyblock **}{key}
	\funcarg{krb5_pointer}{keyseed}
\funcendfuncarg
\funcarg{krb5_pointer}{keyseed}
\funcfuncarg{krb5_error_code}{(*decrypt_proc)}
	\funcarg{krb5_keyblock *}{key}
	\funcarg{krb5_pointer}{decryptarg}
	\funcarg{krb5_kdc_rep *}{dec_rep}
\funcendfuncarg
\funcarg{krb5_pointer}{decryptarg}
\funcinout
\funcarg{krb5_creds *}{creds}
\funcarg{krb5_ccache}{ccache}
\end{funcdecl}

All-purpose initial ticket routine, usually called via
\funcname{krb5_get_in_tkt_with_password} or
\funcname{krb5_get_in_tkt_with_skey}.

Attempts to get an initial ticket for \funcparam{creds{\ptsto}client} to use server
\funcparam{creds{\ptsto}server}, (realm is taken from
\funcparam{creds{\ptsto}client}), with options 
\funcparam{options}, requesting encryption type \funcparam{etype}, and using
\funcparam{creds{\ptsto}times.starttime},  \funcparam{creds{\ptsto}times.endtime},
\funcparam{creds{\ptsto}times.renew_till}
as from, till, and rtime.  \funcparam{creds{\ptsto}times.renew_till} is
ignored unless the RENEWABLE option is requested.

\funcparam{key_proc} is called to fill in the key to be used for decryption.
\funcparam{keyseed} is passed on to \funcparam{key_proc}.

\funcparam{decrypt_proc} is called to perform the decryption of the
response (the encrypted part is in \funcparam{dec_rep{\ptsto}enc_part}; the
decrypted part should be allocated and filled into
\funcparam{dec_rep{\ptsto}enc_part2}.
\funcparam{decryptarg} is passed on to \funcparam{decrypt_proc}.

If \funcparam{addrs} is non-NULL, it is used for the addresses
requested.  If it is null, the system standard addresses are used.

A succesful call will place the ticket in the credentials cache
\funcparam{ccache} and fill in \funcparam{creds} with the ticket
information used/returned.

Returns system errors, encryption errors.

\begin{funcdecl}{krb5_get_in_tkt_with_password}{krb5_error_code}{\funcin}
\funcarg{krb5_flags}{options}
\funcarg{krb5_address **}{addrs}
\funcarg{krb5_enctype}{etype}
\funcarg{krb5_keytype}{keytype}
\funcarg{char *}{password}
\funcarg{krb5_ccache}{ccache}
\funcinout
\funcarg{krb5_creds *}{creds}
\end{funcdecl}


Attempts to get an initial ticket for \funcparam{creds{\ptsto}client} to use server
\funcparam{creds{\ptsto}server}, (realm is taken from
\funcparam{creds{\ptsto}client}), with options 
\funcparam{options}, requesting encryption type \funcparam{etype}, and using
\funcparam{creds{\ptsto}times.starttime},
\funcparam{creds{\ptsto}times.endtime},
\funcparam{creds{\ptsto}times.renew_till}
as from, till, and rtime.  \funcparam{creds{\ptsto}times.renew_till} is
ignored unless the RENEWABLE option is requested.

If \funcparam{addrs} is non-NULL, it is used for the addresses
requested.  If it is null, the system standard addresses are used.

If \funcparam{password} is non-NULL, it is converted using the
cryptosystem entry point for a string conversion routine, seeded with
the client's principal name.  If \funcparam{password} is passed as NULL,
the password is read from the terminal, and then converted into a key.

A succesful call will place the ticket in the credentials cache
\funcparam{ccache}.

Returns system errors, encryption errors.

\begin{funcdecl}{krb5_get_in_tkt_with_skey}{krb5_error_code}{\funcin}
\funcarg{krb5_flags}{options}
\funcarg{krb5_address **}{addrs}
\funcarg{krb5_enctype}{etype}
\funcarg{krb5_keyblock *}{key}
\funcarg{krb5_ccache}{ccache}
\funcinout
\funcarg{krb5_creds *}{creds}
\end{funcdecl}
Similar to \funcname{krb5_get_in_tkt_with_password}.

Attempts to get an initial ticket for \funcparam{creds{\ptsto}client} to use server
\funcparam{creds{\ptsto}server}, (realm is taken from
\funcparam{creds{\ptsto}client}), with options \funcparam{options}, requesting
encryption type \funcparam{etype}, and using 
\funcparam{creds{\ptsto}times.starttime}, \funcparam{creds{\ptsto}times.endtime},
\funcparam{creds{\ptsto}times.renew_till} as from, till, and rtime.
\funcparam{creds{\ptsto}times.renew_till} is ignored unless the
RENEWABLE option is requested.

If \funcparam{addrs} is non-NULL, it is used for the addresses
requested.  If it is null, the system standard addresses are used.

If \funcparam{keyblock} is NULL, an appropriate key for
\funcparam{creds{\ptsto}client} is retrieved from the system key store (e.g.
\filename{/etc/v5srvtab}).  If \funcparam{keyblock} is non-NULL, it is
used as the decryption key.

A succesful call will place the ticket in the credentials cache
\funcparam{ccache}.

Returns system errors, encryption errors.

\begin{funcdecl}{krb5_mk_req}{krb5_error_code}{\funcin}
\funcarg{krb5_principal}{server}
\funcarg{krb5_flags}{ap_req_options}
\funcarg{krb5_checksum *}{checksum}
\funcarg{krb5_ccache}{ccache}
\funcout
\funcarg{krb5_data *}{outbuf}
\end{funcdecl}

Formats a KRB_AP_REQ message into \funcparam{outbuf}.

\funcparam{server} specifies the principal of the server to receive the
message; if credentials are not present in the credentials cache
\funcparam{ccache} for this server, the TGS request with default
parameters is used in an attempt to obtain such credentials, and they
are stored in \funcparam{ccache}.

\funcparam{ap_req_options} specifies the KRB_AP_REQ options desired.

\funcparam{checksum} specifies the checksum to be used in the authenticator.

The \funcparam{outbuf} buffer storage is allocated, and should be freed
by the caller when finished.

Returns system errors.


\begin{funcdecl}{krb5_mk_req_extended}{krb5_error_code}{\funcin}
\funcarg{krb5_flags}{ap_req_options}
\funcarg{krb5_checksum *}{checksum}
\funcarg{krb5_ticket_times *}{times}
\funcarg{krb5_flags}{kdc_options}
\funcarg{krb5_ccache}{ccache}
\funcinout
\funcarg{krb5_creds *}{creds}
\funcout
\funcarg{krb5_data *}{outbuf}
\end{funcdecl}

Formats a KRB_AP_REQ message into \funcparam{outbuf}, with more complete
options than \funcname{krb_mk_req}.

\funcparam{outbuf}, \funcparam{ap_req_options}, \funcparam{checksum},
and \funcparam{ccache} are used in the same fashion as for
\funcname{krb5_mk_req}.

\funcparam{creds} is used to supply the credentials (ticket and session
key) needed to form the request.

If \funcparam{creds{\ptsto}ticket} has no data (length == 0), then a
ticket is obtained from either \funcparam{ccache} or the TGS, passing
\funcparam{creds} to \funcname{krb5_get_credentials}. 
\funcparam{kdc_options} specifies the options requested for the ticket
to be used. If a ticket with appropriate flags is not found in
\funcparam{ccache}, then these options are passed on in a request to an
appropriate KDC.

\funcparam{ap_req_options} specifies the KRB_AP_REQ options desired.

If \funcparam{ap_req_options} specifies AP_OPTS_USE_SESSION_KEY, then
\funcparam{creds{\ptsto}ticket} must contain the appropriate
ENC-TKT-IN-SKEY ticket.

\funcparam{checksum} specifies the checksum to be used in the authenticator.

The \funcparam{outbuf} buffer storage is allocated, and should be freed
by the caller when finished.

Returns system errors.


\begin{funcdecl}{krb5_rd_req_simple}{krb5_error_code}{\funcin}
\funcarg{const krb5_data *}{inbuf}
\funcarg{const krb5_principal}{server}
\funcarg{constkrb5_address *}{sender_addr}
\funcout
\funcarg{krb5_tkt_authent *}{authdat}
\end{funcdecl}

Parses a KRB_AP_REQ message, returning its contents.

\funcparam{server} specifies the expected server's name for the ticket.

\funcparam{sender_addr} specifies the address(es) expected to be present
in the ticket.

A replay cache name derived from the first component of the service name
is used.

The default key store is consulted to find the service key.

\funcparam{authdat{\ptsto}ticket} and
\funcparam{authdat{\ptsto}authenticator} are set to allocated storage
structures; the caller should free them when finished.

Returns system errors, encryption errors, replay errors.


\begin{funcdecl}{krb5_rd_req}{krb5_error_code}{\funcin}
\funcarg{krb5_data *}{inbuf}
\funcarg{krb5_principal}{server}
\funcarg{krb5_address *}{sender_addr}
\funcarg{krb5_pointer}{fetchfrom}
\funcfuncarg{krb5_error_code}{(*keyproc)}
\funcarg{krb5_pointer}{keyprocarg}
\funcarg{krb5_principal}{principal}
\funcarg{krb5_kvno}{vno}
\funcarg{krb5_keyblock **}{key}
\funcendfuncarg
\funcarg{krb5_pointer}{keyprocarg}
\funcinout
\funcarg{krb5_rcache}{rcache}
\funcout
\funcarg{krb5_tkt_authent *}{authdat}
\end{funcdecl}


Parses a KRB_AP_REQ message, returning its contents.

\funcparam{server} specifies the expected server's name for the ticket.

\funcparam{sender_addr} specifies the address(es) expected to be present
in the ticket.

\funcparam{rcache} specifies a replay detection cache used to store
authenticators and server names.

\funcparam{keyproc} specifies a procedure to generate a decryption key for the
ticket.  If \funcparam{keyproc} is non-NULL, \funcparam{keyprocarg} is
passed to it, and the result used as a decryption key. If
\funcparam{keyproc} is NULL, then \funcparam{fetchfrom} is checked; if
it is non-NULL, it specifies a parameter name from which to retrieve the
decryption key.  If \funcparam{fetchfrom} is NULL, then the default key
store is consulted.

\funcparam{authdat{\ptsto}ticket} and
\funcparam{authdat{\ptsto}authenticator} are set to allocated storage
structures; the caller should free them when finished.

Returns system errors, encryption errors, replay errors.

\begin{funcdecl}{krb5_rd_req_decoded}{krb5_error_code}{\funcin}
\funcarg{krb5_ap_req *}{req}
\funcarg{krb5_principal}{server}
\funcarg{krb5_address *}{sender_addr}
\funcarg{krb5_pointer}{fetchfrom}
\funcfuncarg{krb5_error_code}{(*keyproc)}
\funcarg{krb5_pointer}{keyprocarg}
\funcarg{krb5_principal}{principal}
\funcarg{krb5_kvno}{vno}
\funcarg{krb5_keyblock **}{key}
\funcendfuncarg
\funcarg{krb5_pointer}{keyprocarg}
\funcarg{krb5_rcache}{rcache}
\funcout
\funcarg{krb5_tkt_authent *}{authdat}
\end{funcdecl}

Essentially the same as \funcname{krb_rd_req}, but uses a decoded AP_REQ
as the input rather than an encoded input.

\begin{funcdecl}{krb5_mk_error}{krb5_error_code}{\funcin}
\funcarg{krb5_error *}{dec_err}
\funcout
\funcarg{krb5_data *}{enc_err}
\end{funcdecl}

Formats the error structure \funcparam{*dec_err} into an error buffer
\funcparam{*enc_err}.

The error buffer storage is allocated, and should be freed by the
caller when finished.

Returns system errors.

\begin{funcdecl}{krb5_rd_error}{krb5_error_code}{\funcin}
\funcarg{krb5_data *}{enc_errbuf}
\funcout
\funcarg{krb5_error *}{dec_error}
\end{funcdecl}

Parses an error message from \funcparam{enc_errbuf} and fills in the
contents of \funcparam{dec_error}.

Upon return \funcparam{dec_error{\ptsto}client},
\funcparam{dec_error{\ptsto}server}, and
\funcparam{dec_error{\ptsto}text}, if non-NULL, point to allocated
storage which the caller should free when finished.

Returns system errors.

\begin{funcdecl}{krb5_mk_safe}{krb5_error_code}{\funcin}
\funcarg{krb5_data *}{userdata}
\funcarg{krb5_cksumtype}{sumtype}
\funcarg{krb5_keyblock *}{key,}
\funcarg{krb5_fulladdr *}{sender_addr}
\funcarg{krb5_fulladdr *}{recv_addr}
\funcout
\funcarg{krb5_data *}{outbuf}
\end{funcdecl}

Formats a KRB_SAFE message into \funcparam{outbuf}.

\funcparam{userdata} is formatted as the user data in the message.
\funcparam{sumtype} specifies the encryption type; \funcparam{key}
specifies the key which might be used to seed the checksum;
\funcparam{sender_addr} and \funcparam{recv_addr} specify the full
addresses (host and port) of the sender and receiver.  The host portion
of \funcparam{sender_addr} is used to form the addresses used in the
KRB_SAFE message.

The \funcparam{outbuf} buffer storage is allocated, and should be freed by the
caller when finished.

Returns system errors.

\begin{funcdecl}{krb5_rd_safe}{krb5_error_code}{\funcin}
\funcarg{krb5_data *}{inbuf}
\funcarg{krb5_keyblock *}{key}
\funcarg{krb5_fulladdr *}{sender_addr}
\funcarg{krb5_fulladdr *}{recv_addr}
\funcout
\funcarg{krb5_data *}{outbuf}
\end{funcdecl}

Parses a KRB_SAFE message from \funcparam{inbuf}, placing the
integrity-protected user data in \funcparam{*outbuf}.

\funcparam{key} specifies the key to be used for decryption of the message.
 
\funcparam{sender_addr} and \funcparam{recv_addr} specify the full
addresses (host and port) of the sender and receiver.

\funcparam{outbuf} points to allocated storage which the caller should
free when finished.

Returns system errors, integrity errors.

\begin{funcdecl}{krb5_mk_priv}{krb5_error_code}{\funcin}
\funcarg{krb5_data *}{userdata}
\funcarg{krb5_enctype}{etype}
\funcarg{krb5_keyblock *}{key}
\funcarg{krb5_fulladdr *}{sender_addr}
\funcarg{krb5_fulladdr *}{recv_addr}
\funcout
\funcarg{krb5_data *}{outbuf}
\end{funcdecl}

Formats a KRB_PRIV message into \funcparam{outbuf}.

\funcparam{userdata} is formatted as the user data in the message.
\funcparam{etype} specifies the encryption type; \funcparam{key}
specifies the key for the encryption; \funcparam{sender_addr} and
\funcparam{recv_addr} specify the full addresses (host and port) of the
sender and receiver.

The \funcparam{outbuf} buffer storage is allocated, and should be freed by the
caller when finished.

Returns system errors.

\begin{funcdecl}{krb5_rd_priv}{krb5_error_code}{\funcin}
\funcarg{krb5_data *}{inbuf}
\funcarg{krb5_keyblock *}{key}
\funcarg{krb5_fulladdr *}{sender_addr}
\funcarg{krb5_fulladdr *}{recv_addr}
\funcout
\funcarg{krb5_data *}{outbuf}
\end{funcdecl}

Parses a KRB_PRIV message from \funcparam{inbuf}, placing the confidential user
data in \funcparam{*outbuf}.

\funcparam{key} specifies the key to be used for decryption of the message.
 
\funcparam{sender_addr} and \funcparam{recv_addr} specify the full
addresses (host and port) of the sender and receiver.

\funcparam{outbuf} points to allocated storage which the caller should
free when finished.

Returns system errors, integrity errors.

\begin{funcdecl}{krb5_parse_name}{krb5_error_code}{\funcin}
\funcarg{char *}{name}
\funcout
\funcarg{krb5_principal *}{principal}
\end{funcdecl}

Converts a single-string representation \funcparam{name} of the
principal name to the multi-part principal format used in the protocols.

\funcparam{*principal} will point to allocated storage which should be freed by
the caller (using \funcname{krb5_free_principal}) after use.

Returns system errors XXX.

\begin{funcdecl}{krb5_unparse_name}{krb5_error_code}{\funcin}
\funcarg{krb5_principal}{principal}
\funcout
\funcarg{char **}{name}
\end{funcdecl}

Converts the multi-part principal name \funcparam{principal} from the
format used in the protocols to a single-string representation of the name.

\funcparam{*name} points to allocated storage and should be freed by the caller
when finished.

Returns system errors XXX.

\begin{funcdecl}{krb5_address_search}{krb5_boolean}{\funcin}
\funcarg{krb5_address *}{addr}
\funcarg{krb5_address **}{addrlist}
\end{funcdecl}

If \funcparam{addr} is listed in \funcparam{addrlist}, or
\funcparam{addrlist} is null, return TRUE.  If not listed, return FALSE.

\begin{funcdecl}{krb5_address_compare}{krb5_boolean}{\funcin}
\funcarg{krb5_address *}{addr1}
\funcarg{krb5_address *}{addr2}
\end{funcdecl}

If the two addresses are the same, return TRUE, else return FALSE.

\begin{funcdecl}{krb5_principal_compare}{krb5_boolean}{\funcin}
\funcarg{krb5_principal}{p1}
\funcarg{krb5_principal}{p2}
\end{funcdecl}

If the two principals are the same, return TRUE, else return FALSE.

\begin{funcdecl}{krb5_fulladdr_order}{int}{\funcin}
\funcarg{krb5_fulladdr *}{addr1}
\funcarg{krb5_fulladdr *}{addr2}
\end{funcdecl}

Return an ordering on the two full addresses:  0 if the same,
$< 0$ if first is less than 2nd, $> 0$ if first is greater than 2nd.


\begin{funcdecl}{krb5_copy_keyblock}{krb5_error_code}{\funcin}
\funcarg{krb5_keyblock *}{from}
\funcout
\funcarg{krb5_keyblock *}{to}
\end{funcdecl}

Copy a keyblock from \funcparam{from} to \funcparam{to}, including
allocated storage.

\begin{funcdecl}{krb5_copy_creds}{krb5_error_code}{\funcin}
\funcarg{krb5_creds *}{incred}
\funcout
\funcarg{krb5_creds **}{outcred}
\end{funcdecl}

Copy a credentials structure, filling in \funcparam{*outcred} to point
to the newly allocated copy, which should be freed with
\funcname{krb5_free_creds}.

\begin{funcdecl}{krb5_copy_data}{krb5_error_code}{\funcin}
\funcarg{krb5_data *}{indata}
\funcout
\funcarg{krb5_data **}{outdata}
\end{funcdecl}

Copy a data strucutre, filling in \funcparam{*outdata} to point to the
newly allocated copy, which should be freed with \funcname{krb5_free_data}.

\begin{funcdecl}{krb5_copy_principal}{krb5_error_code}{\funcin}
\funcarg{krb5_principal}{inprinc}
\funcout
\funcarg{krb5_principal *}{outprinc}
\end{funcdecl}
Copy a principal structure, filling in \funcparam{*outprinc} to point to
the newly allocated copy, which should be freed with
\funcname{krb5_free_principal}.



\subsection{Credentials cache functions}
The credentials cache functions (some of which are macros which call to
specific types of credentials caches) deal with storing credentials
(tickets, session keys, and other identifying information) in a
semi-permanent store for later use by different programs.

\begin{funcdecl}{krb5_cc_resolve}{krb5_error_code}{\funcinout}
\funcarg{krb5_context}{context}
\funcin
\funcarg{char *}{string_name}
\funcout
\funcarg{krb5_ccache *}{id}
\end{funcdecl}

Fills in \funcparam{id} with a ccache identifier which corresponds to
the name in \funcparam{string_name}.  

Requires that \funcparam{string_name} be of the form ``type:residual'' and
``type'' is a type known to the library.

\begin{funcdecl}{krb5_cc_gen_new}{krb5_error_code}{\funcinout}
\funcarg{krb5_context}{context}
\funcin
\funcarg{krb5_cc_ops *}{ops}
\funcout
\funcarg{krb5_ccache *}{id}
\end{funcdecl}


Fills in \funcparam{id} with a unique ccache identifier of a type defined by
\funcparam{ops}.  The cache is left unopened.

\begin{funcdecl}{krb5_cc_register}{krb5_error_code}{\funcinout}
\funcarg{krb5_context}{context}
\funcin
\funcarg{krb5_cc_ops *}{ops}
\funcarg{krb5_boolean}{override}
\end{funcdecl}

Adds a new cache type identified and implemented by \funcparam{ops} to
the set recognized by \funcname{krb5_cc_resolve}.
If \funcparam{override} is FALSE, a ticket cache type named
\funcparam{ops{\ptsto}prefix} must not be known.

\begin{funcdecl}{krb5_cc_get_name}{char *}{\funcinout}
\funcarg{krb5_context}{context}
\funcin
\funcarg{krb5_ccache}{id}
\end{funcdecl}

Returns the name of the ccache denoted by \funcparam{id}.

\begin{funcdecl}{krb5_cc_default_name}{char *}{\funcinout}
\funcarg{krb5_context}{context}
\end{funcdecl}

Returns the name of the default credentials cache; this may be equivalent to
\funcnamenoparens{getenv}({\tt "KRB5CCACHE"}) with an appropriate fallback.

\begin{funcdecl}{krb5_cc_default}{krb5_error_code}{\funcinout}
\funcarg{krb5_context}{context}
\funcout
\funcarg{krb5_ccache *}{ccache}
\end{funcdecl}

Equivalent to
\funcnamenoparens{krb5_cc_resolve}(\funcparam{context},
\funcname{krb5_cc_default_name},
\funcparam{ccache}).

\begin{funcdecl}{krb5_cc_initialize}{krb5_error_code}{\funcinout}
\funcarg{krb5_context}{context}
\funcarg{krb5_ccache}{id}
\funcin
\funcarg{krb5_principal}{primary_principal}
\end{funcdecl}

Creates/refreshes a credentials cache identified by \funcparam{id} with
primary principal set to \funcparam{primary_principal}.
If the credentials cache already exists, its contents are destroyed.

Errors:  permission errors, system errors.

Modifies: cache identified by \funcparam{id}.

\begin{funcdecl}{krb5_cc_destroy}{krb5_error_code}{\funcinout}
\funcarg{krb5_context}{context}
\funcarg{krb5_ccache}{id}
\end{funcdecl}

Destroys the credentials cache identified by \funcparam{id}, invalidates
\funcparam{id}, and releases any other resources acquired during use of
the credentials cache.  Requires that \funcparam{id} identifies a valid
credentials cache.  After return, \funcparam{id} must not be used unless
it is first reinitialized using \funcname{krb5_cc_resolve} or
\funcname{krb5_cc_gen_new}.

Errors:  permission errors.

\begin{funcdecl}{krb5_cc_close}{krb5_error_code}{\funcinout}
\funcarg{krb5_context}{context}
\funcarg{krb5_ccache}{id}
\end{funcdecl}

Closes the credentials cache \funcparam{id}, invalidates
\funcparam{id}, and releases \funcparam{id} and any other resources
acquired during use of the credentials cache.  Requires that
\funcparam{id} identifies a valid credentials cache.  After return,
\funcparam{id} must not be used unless it is first reinitialized using
\funcname{krb5_cc_resolve} or \funcname{krb5_cc_gen_new}.


\begin{funcdecl}{krb5_cc_store_cred}{krb5_error_code}{\funcinout}
\funcarg{krb5_context}{context}
\funcin
\funcarg{krb5_ccache}{id}
\funcarg{krb5_creds *}{creds}
\end{funcdecl}

Stores \funcparam{creds} in the cache \funcparam{id}, tagged with
\funcparam{creds{\ptsto}client}.
Requires that \funcparam{id} identifies a valid credentials cache.

Errors: permission errors, storage failure errors.

\begin{funcdecl}{krb5_cc_retrieve_cred}{krb5_error_code}{\funcinout}
\funcarg{krb5_context}{context}
\funcin
\funcarg{krb5_ccache}{id}
\funcarg{krb5_flags}{whichfields}
\funcarg{krb5_creds *}{mcreds}
\funcout
\funcarg{krb5_creds *}{creds}
\end{funcdecl}

Searches the cache \funcparam{id} for credentials matching
\funcparam{mcreds}.  The fields which are to be matched are specified by
set bits in \funcparam{whichfields}, and always include the principal
name \funcparam{mcreds{\ptsto}server}.
Requires that \funcparam{id} identifies a valid credentials cache.

If at least one match is found, one of the matching credentials is
returned in \funcparam{*creds}. The credentials should be freed using
\funcname{krb5_free_credentials}.

Errors: error code if no matches found.

\begin{funcdecl}{krb5_cc_get_principal}{krb5_error_code}{\funcinout}
\funcarg{krb5_context}{context}
\funcin
\funcarg{krb5_ccache}{id}
\funcarg{krb5_principal *}{principal}
\end{funcdecl}

Retrieves the primary principal of the credentials cache (as
set by the \funcname{krb5_cc_initialize} request)
The primary principal is filled into \funcparam{*principal}; the caller
should release this memory by calling \funcname{krb5_free_principal} on
\funcparam{*principal} when finished.

Requires that \funcparam{id} identifies a valid credentials cache.

\begin{funcdecl}{krb5_cc_start_seq_get}{krb5_error_code}{\funcinout}
\funcarg{krb5_context}{context}
\funcarg{krb5_ccache}{id}
\funcout
\funcarg{krb5_cc_cursor *}{cursor}
\end{funcdecl}

Prepares to sequentially read every set of cached credentials.
\funcparam{cursor} is filled in with a cursor to be used in calls to
\funcname{krb5_cc_next_cred}.

\begin{funcdecl}{krb5_cc_next_cred}{krb5_error_code}{\funcinout}
\funcarg{krb5_context}{context}
\funcarg{krb5_ccache}{id}
\funcout
\funcarg{krb5_creds *}{creds}
\funcinout
\funcarg{krb5_cc_cursor *}{cursor}
\end{funcdecl}

Fetches the next entry from \funcparam{id}, returning its values in
\funcparam{*creds}, and updates \funcparam{*cursor} for the next request.
Requires that \funcparam{id} identifies a valid credentials cache and
\funcparam{*cursor} be a cursor returned by
\funcname{krb5_cc_start_seq_get} or a subsequent call to
\funcname{krb5_cc_next_cred}.

Errors: error code if no more cache entries.

\begin{funcdecl}{krb5_cc_end_seq_get}{krb5_error_code}{\funcinout}
\funcarg{krb5_context}{context}
\funcarg{krb5_ccache}{id}
\funcarg{krb5_cc_cursor *}{cursor}
\end{funcdecl}

Finishes sequential processing mode and invalidates \funcparam{*cursor}.
\funcparam{*cursor} must never be re-used after this call.

Requires that \funcparam{id} identifies a valid credentials cache and
\funcparam{*cursor} be a cursor returned by
\funcname{krb5_cc_start_seq_get} or a subsequent call to
\funcname{krb5_cc_next_cred}.

Errors: may return error code if \funcparam{*cursor} is invalid.


\begin{funcdecl}{krb5_cc_remove_cred}{krb5_error_code}{\funcinout}
\funcarg{krb5_context}{context}
\funcin
\funcarg{krb5_ccache}{id}
\funcarg{krb5_flags}{which}
\funcarg{krb5_creds *}{cred}
\end{funcdecl}

Removes any credentials from \funcparam{id} which match the principal
name {cred{\ptsto}server} and the fields in \funcparam{cred} masked by
\funcparam{which}.
Requires that \funcparam{id} identifies a valid credentials cache.

Errors: returns error code if nothing matches; returns error code if
couldn't delete.

\begin{funcdecl}{krb5_cc_set_flags}{krb5_error_code}{\funcinout}
\funcarg{krb5_context}{context}
\funcarg{krb5_ccache}{id}
\funcin
\funcarg{krb5_flags}{flags}
\end{funcdecl}

Sets the flags on the cache \funcparam{id} to \funcparam{flags}.  Useful
flags are defined in {\tt <krb5.h>}.

\begin{funcdecl}{krb5_get_notification_message}{unsigned int}{\funcvoid}
\end{funcdecl}

Intended for use by Windows. Will register a unique message type using
\funcname{RegisterWindowMessage} which will be notified whenever the
cache changes. This will allow all processes to recheck their caches.


\subsection{Replay cache functions}
The replay cache functions deal with verifying that AP_REQ's do not
contain duplicate authenticators; the storage must be non-volatile for
the site-determined validity period of authenticators.

Each replay cache has a string ``name'' associated with it.  The use of
this name is dependent on the underlying caching strategy (for
file-based things, it would be a cache file name).  The
caching strategy uses non-volatile storage so that replay
integrity can be maintained across system failures.

\begin{funcdecl}{krb5_auth_to_rep}{krb5_error_code}{\funcinout}
\funcarg{krb5_context}{context}
\funcin
\funcarg{krb5_tkt_authent *}{auth}
\funcout
\funcarg{krb5_donot_replay *}{rep}
\end{funcdecl}
Extract the relevant parts of \funcparam{auth} and fill them into the
structure pointed to by \funcparam{rep}.  \funcparam{rep{\ptsto}client}
and \funcparam{rep{\ptsto}server} are set to allocated storage and
should be freed when \funcparam{*rep} is no longer needed.

\begin{funcdecl}{krb5_rc_resolve_full}{krb5_error_code}{\funcinout}
\funcarg{krb5_context}{context}
\funcarg{krb5_rcache *}{id}
\funcin
\funcarg{char *}{string_name}
\end{funcdecl}

\begin{sloppypar}
\funcparam{id} is filled in to identify a replay cache which
corresponds to the name in \funcparam{string_name}.  The cache is not opened.
Requires that \funcparam{string_name} be of the form ``type:residual''
and that ``type'' is a type known to the library.
\end{sloppypar}

Before the cache can be used \funcname{krb5_rc_initialize} or
\funcname{krb5_rc_recover} must be called.

Errors: error if cannot resolve name.


\begin{funcdecl}{krb5_rc_resolve_type}{krb5_error_code}{\funcinout}
\funcarg{krb5_context}{context}
\funcarg{krb5_rcache *}{id}
\funcin
\funcarg{char *}{type}
\end{funcdecl}

\internalfunc

Looks up \funcparam{type} in the list of knows cache types and if found
attaches the operations to \funcparam{*id} which must be previously
allocated. 

If \funcparam{type} is not found, {\sc krb5_rc_type_notfound} is returned.

\begin{funcdecl}{krb5_rc_register_type}{krb5_error_code}{\funcin}
\funcarg{krb5_context}{context}
\funcarg{krb5_rc_ops *}{ops}
\end{funcdecl}
Adds a new replay cache type implemented and identified by
\funcparam{ops} to the set recognized by
\funcname{krb5_rc_resolve}.  This function requires that a ticket
cache of the type named in 
\funcparam{ops{\ptsto}prefix} has not been previously registered.


\begin{funcdecl}{krb5_rc_default_name}{char *}{\funcin}
\funcarg{krb5_context}{context}
\end{funcdecl}

\begin{sloppypar}
Returns  the name of the default replay cache; this may be equivalent to
\funcnamenoparens{getenv}({\tt "KRB5RCACHE"}) with an appropriate fallback.
\end{sloppypar}

\begin{funcdecl}{krb5_rc_default_type}{char *}{\funcin}
\funcarg{krb5_context}{context}
\end{funcdecl}

Returns the type of the default replay cache.

\begin{funcdecl}{krb5_rc_default}{krb5_error_code}{\funcinout}
\funcarg{krb5_context}{context}
\funcarg{krb5_rcache *}{id}
\end{funcdecl}

This function returns an unopened replay cache of the default type and
default name (as would be returned by \funcname{krb5_rc_default_type}
and \funcname{krb5_rc_default_name}).  Before the cache can be used
\funcname{krb5_rc_initialize} or \funcname{krb5_rc_recover} must be
called.


\begin{funcdecl}{krb5_rc_initialize}{krb5_error_code}{\funcin}
\funcarg{krb5_context}{context}
\funcarg{krb5_rcache}{id}
\funcarg{krb5_deltat}{auth_lifespan}
\end{funcdecl}

Creates/refreshes the replay cache identified by \funcparam{id} and sets its
authenticator lifespan to \funcparam{auth_lifespan}.  If the 
replay cache already exists, its contents are destroyed.

Errors: permission errors, system errors

\begin{funcdecl}{krb5_rc_recover}{krb5_error_code}{\funcin}
\funcarg{krb5_context}{context}
\funcarg{krb5_rcache}{id}
\end{funcdecl}
Attempts to recover the replay cache \funcparam{id}, (presumably after a
system crash or server restart).

Errors: error indicating that no cache was found to recover

\begin{funcdecl}{krb5_rc_destroy}{krb5_error_code}{\funcin}
\funcarg{krb5_context}{context}
\funcarg{krb5_rcache}{id}
\end{funcdecl}

Destroys the replay cache \funcparam{id}.
Requires that \funcparam{id} identifies a valid replay cache.

Errors: permission errors.

\begin{funcdecl}{krb5_rc_close}{krb5_error_code}{\funcin}
\funcarg{krb5_context}{context}
\funcarg{krb5_rcache}{id}
\end{funcdecl}

Closes the replay cache \funcparam{id}, invalidates \funcparam{id},
and releases any other resources acquired during use of the replay cache.
Requires that \funcparam{id} identifies a valid replay cache.

Errors: permission errors

\begin{funcdecl}{krb5_rc_store}{krb5_error_code}{\funcin}
\funcarg{krb5_context}{context}
\funcarg{krb5_rcache}{id}
\funcarg{krb5_donot_replay *}{rep}
\end{funcdecl}
Stores \funcparam{rep} in the replay cache \funcparam{id}.
Requires that \funcparam{id} identifies a valid replay cache.

Returns KRB5KRB_AP_ERR_REPEAT if \funcparam{rep} is already in the
cache.  May also return permission errors, storage failure errors.

\begin{funcdecl}{krb5_rc_expunge}{krb5_error_code}{\funcin}
\funcarg{krb5_context}{context}
\funcarg{krb5_rcache}{id}
\end{funcdecl}
Removes all expired replay information (i.e. those entries which are
older than then authenticator lifespan of the cache) from the cache
\funcparam{id}.  Requires that \funcparam{id} identifies a valid replay
cache.

Errors: permission errors.

\begin{funcdecl}{krb5_rc_get_lifespan}{krb5_error_code}{\funcin}
\funcarg{krb5_context}{context}
\funcarg{krb5_rcache}{id}
\funcout
\funcarg{krb5_deltat *}{auth_lifespan}
\end{funcdecl}
Fills in \funcparam{auth_lifespan} with the lifespan of
the cache \funcparam{id}.
Requires that \funcparam{id} identifies a valid replay cache.

\begin{funcdecl}{krb5_rc_resolve}{krb5_error_code}{\funcinout}
\funcarg{krb5_context}{context}
\funcarg{krb5_rcache}{id}
\funcin
\funcarg{char *}{name}
\end{funcdecl}

Initializes private data attached to \funcparam{id}.  This function MUST
be called before the other per-replay cache functions.

Requires that \funcparam{id} points to allocated space, with an
initialized \funcparam{id{\ptsto}ops} field.

Since \funcname{krb5_rc_resolve} allocates memory,
\funcname{krb5_rc_close} must be called to free the allocated memory,
even if neither \funcname{krb5_rc_initialize} or
\funcname{krb5_rc_recover} were successfully called by the application.

Returns:  allocation errors.


\begin{funcdecl}{krb5_rc_get_name}{char *}{\funcin}
\funcarg{krb5_context}{context}
\funcarg{krb5_rcache}{id}
\end{funcdecl}

Returns the name (excluding the type) of the rcache \funcparam{id}.
Requires that \funcparam{id} identifies a valid replay cache.

\begin{funcdecl}{krb5_rc_get_type}{char *}{\funcin}
\funcarg{krb5_context}{context}
\funcarg{krb5_rcache}{id}
\end{funcdecl}

Returns the type (excluding the name) of the rcache \funcparam{id}.
Requires that \funcparam{id} identifies a valid replay cache.




\subsection{Key table functions}
The key table functions deal with storing and retrieving service keys
for use by unattended services which participate in authentication exchanges.

Keytab routines are all be atomic.  Every routine that acquires
a non-sharable resource releases it before it returns. 

All keytab types support multiple concurrent sequential scans.

The order of values returned from \funcname{krb5_kt_next_entry} is
unspecified.

Although the ``right thing'' should happen if the program aborts
abnormally, a close routine, \funcname{krb5_kt_free_entry},  is provided
for freeing resources, etc.  People should use the close routine when
they are finished.

\begin{funcdecl}{krb5_kt_register}{krb5_error_code}{\funcinout}
\funcarg{krb5_context}{context}
\funcin
\funcarg{krb5_kt_ops *}{ops}
\end{funcdecl}


Adds a new ticket cache type to the set recognized by
\funcname{krb5_kt_resolve}.
Requires that a keytab type named \funcparam{ops{\ptsto}prefix} is not
yet known.

An error is returned if \funcparam{ops{\ptsto}prefix} is already known.

\begin{funcdecl}{krb5_kt_resolve}{krb5_error_code}{\funcinout}
\funcarg{krb5_context}{context}
\funcin
\funcarg{const char *}{string_name}
\funcout
\funcarg{krb5_keytab *}{id}
\end{funcdecl}

Fills in \funcparam{*id} with a handle identifying the keytab with name
``string_name''.  The keytab is not opened.
Requires that \funcparam{string_name} be of the form ``type:residual'' and
``type'' is a type known to the library.

Errors: badly formatted name.
                
\begin{funcdecl}{krb5_kt_default_name}{krb5_error_code}{\funcinout}
\funcarg{krb5_context}{context}
\funcin
\funcarg{char *}{name}
\funcarg{int}{namesize}
\end{funcdecl}

\funcparam{name} is filled in with the first \funcparam{namesize} bytes of
the name of the default keytab.
If the name is shorter than \funcparam{namesize}, then the remainder of
\funcparam{name} will be zeroed.


\begin{funcdecl}{krb5_kt_default}{krb5_error_code}{\funcinout}
\funcarg{krb5_context}{context}
\funcin
\funcarg{krb5_keytab *}{id}
\end{funcdecl}

Fills in \funcparam{id} with a handle identifying the default keytab.

\begin{funcdecl}{krb5_kt_read_service_key}{krb5_error_code}{\funcinout}
\funcarg{krb5_context}{context}
\funcin
\funcarg{krb5_pointer}{keyprocarg}
\funcarg{krb5_principal}{principal}
\funcarg{krb5_kvno}{vno}
\funcarg{krb5_keytype}{keytype}
\funcout
\funcarg{krb5_keyblock **}{key}
\end{funcdecl}

If \funcname{keyprocarg} is not NULL, it is taken to be a
\datatype{char *} denoting the name of a keytab.  Otherwise, the default
keytab will be used.
The keytab is opened and searched for the entry identified by
\funcparam{principal}, \funcparam{keytype}, and \funcparam{vno}, 
returning the resulting key
in \funcparam{*key} or returning an error code if it is not found. 

\funcname{krb5_free_keyblock} should be called on \funcparam{*key} when
the caller is finished with the key.

Returns an error code if the entry is not found.

\begin{funcdecl}{krb5_kt_add_entry}{krb5_error_code}{\funcinout}
\funcarg{krb5_context}{context}
\funcin
\funcarg{krb5_keytab}{id}
\funcarg{krb5_keytab_entry *}{entry}
\end{funcdecl}

Calls the keytab-specific add routine \funcname{krb5_kt_add_internal}
with the same function arguments.  If this routine is not available,
then KRB5_KT_NOWRITE is returned.

\begin{funcdecl}{krb5_kt_remove_entry}{krb5_error_code}{\funcinout}
\funcarg{krb5_context}{context}
\funcin
\funcarg{krb5_keytab}{id}
\funcarg{krb5_keytab_entry *}{entry}
\end{funcdecl}

Calls the keytab-specific remove routine
\funcname{krb5_kt_remove_internal} with the same function arguments.
If this routine is not available, then KRB5_KT_NOWRITE is returned.

\begin{funcdecl}{krb5_kt_get_name}{krb5_error_code}{\funcinout}
\funcarg{krb5_context}{context}
\funcarg{krb5_keytab}{id}
\funcout
\funcarg{char *}{name}
\funcin
\funcarg{unsigned int}{namesize}
\end{funcdecl}

\funcarg{name} is filled in with the first \funcparam{namesize} bytes of
the name of the keytab identified by \funcname{id}.
If the name is shorter than \funcparam{namesize}, then \funcarg{name}
will be null-terminated.

\begin{funcdecl}{krb5_kt_close}{krb5_error_code}{\funcinout}
\funcarg{krb5_context}{context}
\funcarg{krb5_keytab}{id}
\end{funcdecl}

Closes the keytab identified by \funcparam{id} and invalidates
\funcparam{id}, and releases any other resources acquired during use of
the key table.

Requires that \funcparam{id} identifies a keytab.

\begin{funcdecl}{krb5_kt_get_entry}{krb5_error_code}{\funcinout}
\funcarg{krb5_context}{context}
\funcarg{krb5_keytab}{id}
\funcin
\funcarg{krb5_principal}{principal}
\funcarg{krb5_kvno}{vno}
\funcarg{krb5_keytype}{keytype}
\funcout
\funcarg{krb5_keytab_entry *}{entry}
\end{funcdecl}

\begin{sloppypar}
Searches the keytab identified by \funcparam{id} for an entry whose
principal matches \funcparam{principal}, whose keytype matches 
\funcparam{keytype}, and
whose key version number matches \funcparam{vno}.  If \funcparam{vno} is
zero, the first entry whose principal matches is returned.
\end{sloppypar}

Returns an error code if no suitable entry is found.  If an entry is
found, the entry is returned in \funcparam{*entry}; its contents should
be deallocated by calling \funcname{krb5_kt_free_entry} when no longer
needed.

\begin{funcdecl}{krb5_kt_free_entry}{krb5_error_code}{\funcinout}
\funcarg{krb5_context}{context}
\funcarg{krb5_keytab_entry *}{entry}
\end{funcdecl}

Releases all storage allocated for \funcparam{entry}, which must point
to a structure previously filled in by \funcname{krb5_kt_get_entry} or
\funcname{krb5_kt_next_entry}.

\begin{funcdecl}{krb5_kt_start_seq_get}{krb5_error_code}{\funcinout}
\funcarg{krb5_context}{context}
\funcarg{krb5_keytab}{id}
\funcout
\funcarg{krb5_kt_cursor *}{cursor}
\end{funcdecl}

Prepares to read sequentially every key in the keytab identified by
\funcparam{id}.
\funcparam{cursor} is filled in with a cursor to be used in calls to
\funcname{krb5_kt_next_entry}.

\begin{funcdecl}{krb5_kt_next_entry}{krb5_error_code}{\funcinout}
\funcarg{krb5_context}{context}
\funcarg{krb5_keytab}{id}
\funcout
\funcarg{krb5_keytab_entry *}{entry}
\funcinout
\funcarg{krb5_kt_cursor *}{cursor}
\end{funcdecl}

Fetches the ``next'' entry in the keytab, returning it in
\funcparam{*entry}, and updates \funcparam{*cursor} for the next
request.  If the keytab changes during the sequential get, an error is
guaranteed.  \funcparam{*entry} should be freed after use by calling
\funcname{krb5_kt_free_entry}.

Requires that \funcparam{id} identifies a valid keytab.  and
\funcparam{*cursor} be a cursor returned by
\funcname{krb5_kt_start_seq_get} or a subsequent call to
\funcname{krb5_kt_next_entry}.

Errors: error code if no more cache entries or if the keytab changes.

\begin{funcdecl}{krb5_kt_end_seq_get}{krb5_error_code}{\funcinout}
\funcarg{krb5_context}{context}
\funcarg{krb5_keytab}{id}
\funcarg{krb5_kt_cursor *}{cursor}
\end{funcdecl}

Finishes sequential processing mode and invalidates \funcparam{cursor},
which must never be re-used after this call.

Requires that \funcparam{id} identifies a valid keytab  and
\funcparam{*cursor} be a cursor returned by
\funcname{krb5_kt_start_seq_get} or a subsequent call to
\funcname{krb5_kt_next_entry}.

May return error code if \funcparam{cursor} is invalid.




\subsection{Operating-system specific functions}
The operating-system specific functions provide an interface between the
other parts of the {\tt libkrb5.a} libraries and the operating system.


\section{Principal database functions}

\input{kdb.tex}

\section{Encryption system interface}
\input{encrypt.tex}

\section{Checksum interface}
\input{cksum.tex}

\section{CRC-32 checksum functions}
\input{crc-32.tex}

\appendix
\cleardoublepage
\input{\jobname.ind}
\end{document}
